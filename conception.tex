\section{Conception}
\subsection{Partie 1}
Pour faire clignoter la LED, il suffit de faire un XOR sur la bonne addresse du registre P1OUT. Il n'y a pas eu de difficultés majeures sur cette étape.

\lstinputlisting{Code1.c}

\subsection{Partie 2}
Pour la seconde partie il a fallu créer une interruption afin de pouvoir changer l'état des LEDs 1 à 3.\\
Pour ce faire, il faut tout d'abord préparer le bouton en input, ensuite activer la résistance interne, la mettre en "pull-up" et faire en sorte qu'elle détecte correctement les flancs en modifiant le registre approprié.
Il faut ensuite autoriser les interruptions sur les deux entrée des boutons, puis autoriser les interruptions de manières générale.\\
L'étape suivante est d'écrire les fonctions d'interruptions qui vont agir sur les registres. Il faut les précéder de la directive \#pragma afin d'indexer les registres des ports 1 et 2.
Il ne faut pas oublier de remettre à zéro le bit du registre contenant le flag d'interruption sinon il n'est pas possible de continuer l'exécution du programme principal.

La principale difficulté que j'ai rencontré dans cette étape de la conception était l'oubli de la résistance interne rendant le bouton inutilisable pendant quelques secondes après chaque appui, le temps que la tension se dissipe.
Un autre difficulté fût d'implémenter un anti-rebond pour ne pas prendre un compte les appuis "parasites" sur le bouton.

\lstinputlisting{Code2.c}

\subsection{Partie 3}
Cette partie de la conception s'est révélée relativement similaire à la première. La principale différence se situtant dans la fonction d'interrution.

\lstinputlisting{Code3.c}

\subsection{Partie 4}
Il s'agit du même code que pour la partie 2 mais avec un seul bouton.

\lstinputlisting{Code4.c}
