\section{Conclusion}
En conclusion, il est pratique de pouvoir utiliser les interruptions pour effectuer des actions simples sans perturber outre mesure la boucle principale.
Il faut toutefois rester prudent et s'assurer que le code contenu dans ces fonctions est simple et rapide à exécuter.
Il faut aussi être prudent concernant les boutons, ils sont sujets à pleins de défauts (notamment les rebonds) et il n'est pas aisé de s'en débarasser. Une meilleure solution que celle proposée plus tôt pourrait être d'utiliser des timers.
La partie 3 du labo pourrait aussi être améliorée les LEDs pouvant attendre un peu avant de s'éteindre, l'implémentation actuelle étant limitée par la vitesse de la boucle principale.
