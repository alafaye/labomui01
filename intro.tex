\section*{Introduction}

\subsection*{Introduction}
Ce labo a pour but de tester les possibilites offertes par les interruptions dans le cadre de la conception de programmes pour micro-controleurs.
Pour ce travail pratique, une MSP430F5529 a été utilisée, le test est le debug effectué au travers de son interface JTAG.
Le code a été écrit et testé sous CCS 5.5 avec un Windows 8.1.

\subsection{Buts}
Ce laboratoire est séparé en quatre partie:
\begin{itemize}
\item Tout d'abord changer l'état de la LED PAD5 toutes les demi-secondes
\item Ensuite, tout en laissant la LED clignoter, faire changer l'état des LEDs PAD[1,2,3] avec le bouton S1 ou S2
\item Recopier l'état du bouton S1 sur les LEDs PAD[1,2,3]
\item Finalement faire changer l'état des LEDs PAD[1,2,3] en appuyant sur le bouton S1
\end{itemize}
